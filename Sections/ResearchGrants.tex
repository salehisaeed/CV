

% ===============================================
% \newpage
\section{Research Grants}{}
\begin{itemize}[leftmargin=*,topsep=5pt]
    \item Artificial intelligence for enhanced hydraulic turbine lifetime
    \begin{description}[nosep]
    % \vspace*{-0.4em}
    \item[Funding agency:] Swedish Hydropower Centre (SVC)
    \item[Time period:] from 2023-01 to 2027-06 (four and half years)
    \item[Amount:] 7\,411\,000 SEK (around 655\,000 Euros)
    \item[Role:] Co-PI (PI: Prof. Håkan Nilsson)
    \item[Description:] This is the successful grant that I received for my current position as a researcher at Chalmers Industriteknik and Chalmers University of Technology. I developed the idea for the project and had the leading role in writing the grant application. A requirement for the call is also that the university provides 39.5\% in-kind to the project. Therefore, the grant also includes funding for a PhD student (as in-kind) that we recently hired and the total budget sums up to 12\,498\,000 SEK. %For more information about the grant, please see the attached document in Appendix~\ref{grant:AFC}.
    \end{description}

    \newpage
    \item Hydropower operation and lifetime analysis
    \begin{description}[nosep]
    % \vspace*{-0.4em}
    \item[Funding agency:] Swedish Hydropower Centre (SVC)
    \item[Time period:] from 2023-03 to 2027-08 (four and half years)
    \item[Amount:] 5\,266\,000 SEK (around 465\,000 Euros)
    \item[Role:] Co-PI (PI: Prof. Håkan Nilsson)
    \item[Description:] The funding for this successful grant application was used to hire a PhD student (Martina) whom I am co-supervising. I had a major role in developing the application as it can be considered a continuation of my postdoc project.% and the PhD started basically started where I left off.
    \end{description}

    % \newpage
    \item Multi-Fidelity Physics-Informed Neural Network to Solve Partial Differential Equations
    \begin{description}[nosep]
    % \vspace*{-0.4em}
    \item[Funding agency:] Chalmers University of Technology  (internal call, in competition)
    \item[Time period:] from 2023-01 to 2027-06 (four and half years)
    \item[Amount:] 5\,517\,000 SEK (around 487\,000 Euros)
    \item[Role:] Co-PI (A collaborative application, i.e., no PI or main applicant)
    \item[Description:] This is the funding that we received for a cross-divisional PhD student in the Department of Mechanics and Maritime Sciences of Chalmers University of Technology. I played a pivotal role in formulating the concept and preparing the grant proposal application. Subsequently, the department approved our proposal, leading to the recruitment of a PhD student (Mohammad), whom I am co-supervising. %For more information about the grant and my contribution, please see the attached document in Appendix~\ref{grant:PINN}.

    \end{description}
    
\end{itemize}
% ===============================================




% % ===============================================
% % \newpage
% \section*{Independent Line of Research}

% My ongoing research is related to the combination of artificial intelligence, data science, and computational fluid dynamics, with a specific emphasis on the application of Deep Reinforcement Learning (DRL) in fluid flows. This represents a significant departure from my PhD and postdoctoral work. My current line of research is entirely different than what all my previous supervisors are doing. It is worth highlighting that I developed the idea for the current research and had the leading role in writing the grant application, which subsequently secured funding for my current position.

% During my postdoc position, I have independently supervised a few master's students in a project related to my own area of interest, specifically, unsupervised machine learning and reduced order modeling algorithms for computational fluid dynamics.

% I have also initiated a collaboration with Wengang Mao (professor at the Division of Marine Technology). Our joint efforts resulted in our successful grant application for a cross-divisional PhD project on physics-informed machine learning for fluid mechanics. This collaboration underscores my commitment to promoting interdisciplinary research and leveraging expertise across divisions to advance innovative projects.

% Although the Division of Fluid Dynamics at Chalmers University of Technology has a wide range of expertise in the fields of fluid flows, there are currently no research activities related to my focus on the integration of artificial intelligence and data science with fluid dynamics and heat transfer. My objective is to employ the strength of my multi-faceted background -- with experience in fluid dynamics, machine learning, and uncertainty quantification -- to develop research activities related to the coupling of artificial intelligence and fluid mechanics.
% % ===============================================


